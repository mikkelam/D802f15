\section{Related Work}\label{sec:relatedwork}

Kevin C. and Daniel P have trained a classifier to predict the winning team of a Dota 2 match\cite{dota2article}, by only considering the heroes chosen at the beginning of the game. Dota 2 is a PC game very similar to LoL, also with 5 players on each team.
Each feature used represented the presence or absence of a particular hero on one of the teams.
When training on 18,000 matches they could correctly predict the outcome of 69.8 \% of the 5,669 matches in their test set.
With 50,000 training samples, they almost reached 70 \% correct predictions.

Only matches between high skilled players were considered.

We think the method can they used can be further improved, since they only tried to capture the strength of each hero individually.
They did not consider that some combinations of heroes are stronger when fighting together compared to other combinations.
We believe these synergies are important to consider, which are discussed further in section \ref{sec:choosingfeatures}.

LoL includes 123 different champions in Patch 5.3.0.291. If we want to take account for synergies between that many champions, we may need to train on a huge amount of data, in order to have all combinations of champions represented an adequately number of times in the training samples. We will therefore approach this problem as a big data problem.
Hadoop MapReduce and different variants have successfully been used for big data problems, by distributing the work load between many nodes in a cluster\cite{ApacheSpark}.
In cases where it is only necessary to read from data, the Apache Spark framework seems to be much faster.
For iterative machine learning jobs, it outperforms MapReduce by a factor of 10.

%%% Local Variables:
%%% mode: latex
%%% TeX-master: "../main"
%%% End: