\section{Choosing features}\label{sec:choosingfeatures}
Throughout this article, $T = \{0, 1\}$ denotes the two teams \emph{blue} and \emph{red} respectively, and $C = \{0, 1, \dots, m-1\}$ denotes the set of all champions. When referring to features, $t$ denotes an element in $T$ and $c$ an element in $C$.

Some champions may be better than other champions. To capture the strength of individual champions, we will need a feature that represent the presence or absence of a particular champion on each team.
We define a set of features $X_1 = \{x_1(t, c)\}$, where
\[
x_1(t, c) = 
\begin{cases} 
  1 & \text{if } c \text{ is present on team } t \\
  0 & \text{otherwise} 
\end{cases}
\]

Some champions are considered damage dealers, they deal very high damage, but die easily. Other champions deal very little damage, but can be almost impossible to kill, these are considered tanks. These two types of champions are weak when alone, but when they team up, they can pose a serious threat. The tank can be used by the damage dealer as a living shield, allowing him to stay alive for much longer, thus deal more damage.
To capture the synergy between two champions on the same team, we will for each team need a feature that represents the presence or absence of every 2-combination %Skal muligvis ændres hvis vi har mere end 2kombinationer senere //Funder 
of champions on that team. We therefore define a set of features $X_2 = \{x_2(t, c_1, c_2)\}$, where $c_1 < c_2$, and
\[
x_2(t, c_1, c_2) = 
\begin{cases} 
1 & \text{if both } c_1 \text{ and } c_2 \text{ are present on team } t \\
0 & \text{otherwise} 
\end{cases}
\]

We make the restriction $c_1 < c_2$, because we want to ignore permutations. This is because the two features $x_2(t, c_1, c_2)$ and $x_2(t, c_2, c_1)$ represent the same thing, namely that the two champions $c_1$ and $c_2$ both are present on team $t$.

Some champions may have an advantage when fighting against a particular opposing champion.
For instance, a champion that is good at dodging ranged attacks is good against an enemy that only has ranged attacks.
We say that the better suited champion \emph{counters} the other.
To capture that one champion may counter another, we will for each champion on team $t$ need a feature that represents the presence or absence of every possible champion on the other team.
For all $c_1, c_2 \in C$, we define a set of features $X_3= \{x_3(c_1, c_2) \}$, where
\[x_3(c_1, c_2) = 
\begin{cases} 
1 & \text{if } c_1 \text{ is present on team blue and } c_2 \text{ is present on team red} \\ 
0 & \text{otherwise} 
\end{cases}\]

In this case, we do not have the restriction $c_1, c_2$ thus consider permutations instead of combinations.
To understand why, consider that $c_1$ counters $c_2$.
In this case, the feature $x_3(c_1, c_2) = 1$ is favorable to the blue team, while $x_3(c_2, c_1) = 1$ is favorable to the red team.
Note that in some game modes, $x_3(c_1, c_2) = x_3(c_2, c_1) = 1$ is allowed. That is, the same champions may appear on both teams.
For this type of feature, we may also have that $c_1 = c_2$.
In that way we can capture if a champion can counter itself due to some asymmetries in the map layout.

%%% Local Variables:
%%% mode: latex
%%% TeX-master: "../main"
%%% End: