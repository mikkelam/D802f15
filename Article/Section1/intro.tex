\section{Introduction}\label{sec:intro}

Online multiplayer games have the possibiltiy of generating a lot of data, this only increases with the number of players and possible options for affecting the power of the player or the team. 
League of Legends, created by Riot games, was the most played online game in the beginning of 2015~\cite{LoLmostplayed}, mustering 27 million people playing it daily, in the beginning of 2014~\cite{LoL27mill}. 
When playing the game, the players are divided into two teams of 5 players each, from a selection of more than 100 different champions, each player picks a single one. 
Each team then starts at their respective base, either as the red team in the upper right or as the blue team in the lower left corner. The goal is now to destroy the nexus of the opposing team and protect your own nexus. There are three paths connecting the two bases and both bases with a total of four towers, two belonging to each team. Each base will also send out small packs of minions to assist the champions in the destruction of the towers and eventually the nexus.
As the game advances experience and money is earned depending on the performance of the players ans the team as a whole. The experience is used to improve the skills of the champion while money is spend purchasing items that will help the player or the team. 
The data Riot games gather from the game is extensive with many parameters~\cite{LoLparameters}, which gives it many possible interesting applications. 

This paper will use the data to create a model capable of predicting the winning team based on the composition of champions, or even predicting which champion should be chosen to increase the probability of winning, given the opponents selection of champions. Since there are so many different champions and purchasable items, the variations of possible options the player can take is large, this means that we will require a great deal of data, more than can be stored optimally in any given database. We will require a cluster to handle this newly created big data problem and all its computations.


%%% Local Variables:
%%% mode: latex
%%% TeX-master: "../main"
%%% End:
