\section{Classification}\label{sec:classification}
In this section we present alternative classification models that can be used instead of logistical regression.
% intro

% features - target(class)

\subsection{Linear Regression}\label{sec:linreg}
% linear regression
The goal of linear regression is to fit a linear function to a set of input-output pairs. 
If the input features are $X_1,X_2 \dots, X_n$ and weights $\overline{w} = w_0, w_1, \dots, w_n$.

\[f^{\overline{w}}(X_1,X_2, \dots, X_n) = w_0 + w_1 X_1 + w_2 X_2 + \cdots + w_n X_n \]

We will learn the function for each target feature $Y$ separately, so given a set $E$ of examples.
The function $val(e,X_i)$ give us the value of the feature $X_i$ in example $e \in E$. The function 
\begin{align*}
pval^{\overline{w}}(e,Y) &= w_0 + w_1 \text{ val}(e,X_1) + w_2 \text{ val}(e,X_2) + \cdots + w_n \text{ val}(e,X_n) \\
&= \sum^n_{i=0} w_i \text{ val}(e,X_i)
\end{align*}
give us the predicted value for the target feature $Y$ for the training example $e$.
In order to avoid having a special case we assume a feature $X_0$ with $val(e,X_0)=1$.~\cite[p. 304]{AI2010}


\subsection{Cost Function}\label{sec:costfun}
% cost function
A cost function is a measure on how much it will ``cost'' if the prediction for a given example is wrong.
Three basic cost functions exist: the absolute error, the sum-of-squares error and 
the worst-case error.

The absolute error function provides the sum of all absolute errors on the examples in $E$.

\[AbsErr = \sum_{e \in E}\sum_{Y \in T} |val(e,Y) - pval(e,Y)|\]

This function is always non-negative and only zero when $Y$ is predicted correctly in all examples. 

The sum-of-squares error function assigns a larger error the further a prediction is from the actual value.

\[ SSErr = \sum_{e \in E}\sum_{Y \in T} (val(e,Y) - pval(e,Y))^2 \]

The worst-case error function gives the maximum error predicted for all examples.

\[ MaxErr = \max_{e \in E}\max_{Y \in T} | val(e,Y) - pval(e,Y) | \]

\begin{flushright}
\cite[p. 290-291]{AI2010}
\end{flushright}

Using the sum-of-squares error function we get the cost function.

\begin{align*}
Error_E(\overline{w}) &= \sum_{e \in E} \left(val(e,Y)-pval^{\overline{w}}(e,Y)\right)^2 \\
&= \sum_{e \in E} \left(val(e,Y)-\sum_{i=0}^n w_i \times val(e,X_i)\right)^2
\end{align*}

The goal when learning a classification model, is to chose the weights that minimise the error.

\[ \text{argmin}_{\overline{w}} \text{ Error}_E(\overline{w}) \]

In the linear case these weights can be found analytically by taking the partial derivative of the cost function with respect 
to each weight, setting it equal to zero and solving for the weight.
However when using a more complex model or a large set of features, a more generel iterative solution is required.

\subsubsection{Gradient descent}

Another way of finding the weights that minimise the cost function, is called gradient descent.
This is an iterative approach that in each step takes the decreases of the weights in
proportion to their partial derivative.
\[ w_i = w_i - \eta \times \frac{\partial \text{ Error}_E(\overline{w})}{\partial w_i} \]
where $\eta$ is the learning rate.

For the sum-of-squares error function we get

\begin{align*}
w_i &:= w_i - \eta \times \frac{\partial \sum_{e \in E}\left(val(e,Y) - pval^{\overline{w}}(e,Y)\right)^2}{\partial w_i} \\ %Kolon? //Funder
&= w_i - \eta \times -2 \times \left(val(e,Y) - pval^{\overline{w}}(e,Y)\right) \times val(e,X_i) \\
&= w_i + \eta \times \delta \times val(e,X_i)
\end{align*}

Where $\delta = val(e,Y) - pval^{\overline{w}}(e,Y)$ and the constant 2 has been absorbed into $\eta$.

\begin{flushright}
\cite[p. 305]{AI2010}
\end{flushright}


%%% Local Variables:
%%% mode: latex
%%% TeX-master: "../main"
%%% End:
