\subsection{Choosing features}\label{sec:choosingfeatures}
In this section, we aim to define the features $\phi_j(x)$ for each LoL match $x$, as described in \Cref{sec:phi}.
Each feature is extracted by a mapping function, that maps a LoL match to a binary value, that is either $0$ or $1$.
Before the feature mappings are defined, convenient notations are introduced that mathematically describes the concept of a game.
\begin{itemize}

\item $\text{CHAMPIONS}$ is the set of all champions, each represented by an id. $|\text{CHAMPIONS}| = 124$ in the patch version of LoL used in this project.
\item $\text{PLAYERS}$ is the set of all players.
\item $\text{TEAMS} = \{\textit{blue}, \textit{purple}\}$ are the two teams.
\item $T_i(x) = \{ p \in \text{PLAYERS} \mid p \text{ plays on team } i \text{ in match } x \}$, where $i \in \text{TEAMS}$.
\item $\text{RANKS} = \{\textit{unranked},\textit{bronze},\textit{silver},\textit{gold},\textit{platinum},\textit{diamond},\textit{master},\textit{challenger}\}$ is the set of ranks.
\item $\text{rank}(p) : \text{PLAYERS} \rightarrow \text{RANKS}$ returns the rank of player $p$.
\item $\text{LANES} = \{\textit{top},\textit{bottom},\textit{mid},\textit{jungle}\}$ is the set of lanes.
\item $\text{SPELLS}$ is the set of summoner spells. $|\text{SPELLS}| = 22$.
\item $\text{RUNES}$ is the set of runes. $|\text{RUNES}| = 296$.
\item $\text{MASTERIES}$ is the set of masteries. $|\text{MASTERIES}| = 57$.
\end{itemize}

Many features are very similar. Therefore, we group features by similarity.
In the following, 10 different types of features are defined. Most types of features take additional parameters to index specific features within that group.
Remember that the definition of every feature is with respect to the match $x$ taken as input.

\subsubsection{Single champions}
Some champions may be better than others. To capture the strength of individual champions, we define a features that represent the presence or absence of a particular champion on each team.
$\forall t \in \text{TEAMS}, \forall c \in \text{CHAMPIONS}:$
\begin{equation}\label{eq:single}  
\phi_{\text{SINGLE}, t, c}(x) = 
\begin{cases} 
  1 & \text{if } c \text{ is on team } t\\
  0 & \text{otherwise} 
\end{cases}
\end{equation}

\subsubsection{Champion pairs}
Some champions are considered damage dealers. They deal very high damage, but die easily. Other champions deal very little damage, but can be almost impossible to kill, these are considered tanks. These two types of champions are weak when alone, but when they team up, they can pose a serious threat. The tank can be used by the damage dealer as a living shield, allowing him to stay alive for much longer, thus deal more damage.
To capture the synergy between two champions on the same team, we define features that represent the presence or absence of every 2-combination of champions on that team. $\forall t \in \text{TEAMS}, \forall c_1, c_2 \in \text{CHAMPIONS}$ where $c_1 < c_2$:
\begin{equation}\label{eq:pair}
\phi_{\text{PAIR}, t, c_1, c_2}(x) =
\begin{cases}
  1 & \text{if both } c_1 \text{ and } c_2 \text{ are on team } t\\
  0 & \text{otherwise}
\end{cases}
\end{equation}

We make the restriction $c_1 < c_2$, because we want to ignore permutations. This is because the two features $x_\text{PAIR}(t, c_1, c_2)$ and $x_\text{PAIR}(t, c_2, c_1)$ are the same, since they both capture that the two champions $c_1$ and $c_2$ are present on team $t$.

\subsubsection{Champion counters}
Some champions may have an advantage when fighting against a particular opponent.
For instance, a champion that is good at dodging ranged attacks is good against an enemy that only has ranged attacks.
We say that the better suited champion \emph{counters} the other.
To capture that one champion may counter another, we will for each champion on team $t$ need a feature that represents the presence or absence of every possible champion on the opposing team. $\forall c_1, c_2 \in \text{CHAMPIONS}:$
\begin{equation}\label{eq:counter}
\phi_{\text{COUNTER},c_1,c_2}(x) = 
\begin{cases} 
1 & \text{if } c_1 \text{ is on blue team and } c_2 \text{ is on purple team} \\ 
0 & \text{otherwise} 
\end{cases}
\end{equation}

For this type of feature, we do not have the restriction $c_1 < c_2$ and thus consider permutations instead of combinations.
To understand why, consider that $c_1$ counters $c_2$.
In this case, the feature $\phi_{\text{COUNTER},c_1,c_2}(x) = 1$ is favorable to the blue team, while $\phi_{\text{COUNTER},c_2,c_1}(x) = 1$ is favourable to the purple team.
Note that in some game modes, $\phi_{\text{COUNTER},c_1,c_2}(x) = 1$, is allowed for $c_1 = c_2$. That is, the same champion may appear on both teams.
In that way we can capture if a champion can counter itself due to some asymmetries in the map layout.

\subsubsection{Player ranks}
In LoL, players can compete in ranked games, where they are placed in one of 7 tiers. Better players achieve higher tiers.
Before a match starts, we have access to data about the highest tier each player has achieved in a ranked game, which we will refer to as the rank of a player. We also know if a player has not competed in ranked games, in which case we say that he is unranked.
A player can achieve one of the ranks \textit{bronze}, \textit{silver}, \textit{gold}, \textit{platinum}, \textit{diamond}, \textit{master}, \textit{challenger}, mentioned in increasing order of skills required to achieve the rank.
We define a score function $\mu : \text{PLAYERS} \rightarrow \mathcal{N}$, where $\mu(p) = 0$ if $p$ is \textit{unranked}, or $1, 2, \dots, 7$ if $p$ has rank \textit{bronze}, \textit{silver}, $\dots$, \textit{challenger} respectively.
We define the rank of team $t$ in a match $x$, to be the average rank of all players on that team who is not unranked:
\begin{equation}\label{eq:eta}
\eta(t, x) = \frac{\sum\limits_{p \in T_t(x)} \mu(p)}{|\{p \in T_t(x) \mid \mu(p) > 0\}|}
\end{equation}
Note that if all players are \textit{unranked}, we exceptionally define $\eta(t, x) = 0$. We use the definition of rank to define 3 new features that capture whether blue, purple, or no team has the highest rank:
$\forall t \in \text{TEAMS}$
\begin{equation}\label{eq:bestrank}
\phi_{\text{BEST-RANK},t}(x) = 
\begin{cases} 
  1 & \text{if } \eta(t,x) > \eta(t',x) \text{ for all } t' \in \text{TEAMS where } t \neq t'\\
  0 & \text{otherwise} 
\end{cases}  
\end{equation}

Additionally, we define a feature to capture a tie between the rank of the two teams:
\begin{equation}\label{eq:bestrank}
\phi_{\text{BEST-RANK},\text{none}}(x) = 
\begin{cases} 
  1 & \text{if } \phi_{\text{BEST-RANK},t}(x) = 0 \text{ for all } t \in \text{TEAMS}\\
  0 & \text{otherwise} 
\end{cases}  
\end{equation}

The $\phi_\text{BEST-RANK}$ feature may have some shortcomings. The assumption that the rank of each player can be mapped to a score on a linear scale may not be entirely on spot.
Also, it may be easier to predict one team to be a winner, if the average rank of the two teams are considerably different. That is, if the players on one of the team have much greater ranks in average. The difference between the average rank of each team is not captured by the $\phi_\text{BEST-RANK}$ feature.
Therefore, another type of feature is introduced that captures the exact rank of all players on each team.
$\forall t \in \text{TEAMS}, \forall i \in \{1,2,\dots,5\}, \forall r \in \text{RANKS}$:
\begin{equation}\label{eq:playerrank}
\phi_{\text{PLAYER-RANK}i,t,r}(x) = 
\begin{cases} 
  1 & \text{if the } i \text{'th player on team } t \text{ has rank } r\\
  0 & \text{otherwise} 
\end{cases}  
\end{equation}

In the early stage of a match, players tend to keep their champion in the same lane.
If a good player plays against a bad player in the same lane, the good player might get so strong that he can carry the team to victory.
Therefore it may be good to have a feature that considers the rank of players who fights in the same lane.
Before defining a feature that considers the rank of players that play against each other in the same lane,
we first define a function that given a match, a lane, and a team, returns the rank(s) of the players(s) playing that lane on the given team.
$\forall l \in \text{LANES}, \forall t \in \text{TEAMS}$ and a given match $x$:
\begin{equation}\label{eq:xi}
  \xi(t,l,x) =
\begin{cases} 
  \text{none} & \text{if no players on team } t \text{ plays lane } l \text{.}\\
  \text{rank}(p) & \text{if some } p \text{ is the only player playing}\\
  & \text{lane } l \text{ on team } t.\\
  (\text{rank}(p_1), \text{rank}(p_2)) & \text{if some } p_1, p_2 \text{ are the only players playing}\\
                        & \text{lane }l \text{ on team } t \text{, and } p_1 \neq p_2.\\
  \text{many} & \text{otherwise}.
\end{cases}
\end{equation}
Note that if more than two players on the same team plays the same lane, $\xi$ does not capture any ranks for that team and lane. This is because of the huge amount of combinations and the fact that playing three champions in the same lane is a very bad strategy. Therefore, we think a single feature is enough to capture all the bad strategies of having three champions in the same lane.
 
The function $\xi$ is now used to define a new feature that captures the rank of players playing in each lane.
$\forall l \in \text{LANES}$, and for any $a$ and $b$ that are possibles values in the range of $\xi$:
\begin{equation}\label{eq:laneranks}
\phi_{\text{LANE-RANKS},l,a,b}(x) =
\begin{cases} 
  1 & \text{if } \xi(\textit{blue},l,x) = a, \xi(\textit{purple},l,x) = b\\
  0 & \text{otherwise} 
\end{cases}  
\end{equation}

\subsubsection{Champion lanes}
Some champions are better suited for a particular lane. For instance, mages tend to go mid as they generally lose both mana and health fast, and the path back to the base for regeneration is shortest from mid. To capture the lane each champion play, we define a new feature.
$\forall c \in \text{CHAMPIONS}, \forall l \in \text{LANES}, \forall t \in \text{TEAMS}$:
\begin{equation}\label{eq:championlane}
  \phi_{\text{LANE-CHAMPION},c,l,t}(x) =
\begin{cases}
  1 & \text{if champion } c \text{ fought at lane } l \text{ for team } t\\
  0 & \text{otherwise}
\end{cases}
\end{equation}

\subsubsection{Champion spells}
Some summoner spells are better suited for particular champions. For instance, a healing or shielding spell is good for the high damaging champions that die easily.
To capture the spells used by the different summoners, we define a new feature:
$\forall c \in \text{CHAMPIONS}, \forall s \in \text{SPELLS}, \forall t \in \text{TEAMS}$:
\begin{equation}\label{eq:championspell}
  \phi_{\text{CHAMPION-SPELL},c,s,t}(x) =
\begin{cases} 
  1 & \text{if champion } c \text{ uses spell } s \text{ on team } t\\
  0 & \text{otherwise} 
\end{cases}
\end{equation}

\subsubsection{Champion runes}
Some runes are better suited for particular champions. For instance, champions that deal physical damage will benefit more from having runes that increase physical damage instead of magic damage. To capture the runes used by the different champions, we define a new feature:
$\forall c \in \text{CHAMPIONS}, \forall r \in \text{RUNES}, \forall t \in \text{TEAMS}$:
\begin{equation}\label{eq:championrunes}
  \phi_{\text{CHAMPION-RUNE},c,r,t}(x) =
\begin{cases} 
  1 & \text{if champion } c \text{ uses rune } r \text{ on team } t\\
  0 & \text{otherwise} 
\end{cases}
\end{equation}

\subsubsection{Champion masteries}
With same reasoning as for runes, some masteries are better suited for particular champions. To capture the runes used by the different champions, we define a new feature:
$\forall c \in \text{CHAMPIONS}, \forall m \in \text{MASTERIES}, \forall t \in \text{TEAMS}$:
\begin{equation}\label{eq:championmastery}
  \phi_{\text{CHAMPION-MASTERY},c,m,t}(x) =
\begin{cases} 
  1 & \text{if champion } c \text{ uses mastery } m \text{ on team } t\\
  0 & \text{otherwise} 
\end{cases}
\end{equation}

\subsection{Feature sparsity}\label{sec:featuresparsity}
All of the introduced features can have value of either 1 or 0. If a feature has a value of $1$ in a given match, we say that the given feature \emph{appear} in that match.
For a given match, a dense representation of features is a list of values, such that the $i$'th value denotes the value of feature $i$.
A sparse representation is a list of indices $i$ of all features that appear in the given match.
If there in general only appear very few of the possible features in any given match, a sparse representation may be more memory efficient and hence provide better performance and thus be favourable.

\begin{description}
\item[{$\phi_{\text{SINGLE}}$}]\hfill
 
Each team can choose from $|\text{CHAMPIONS}|$ different champions. Therefore:
    \[|\phi_{\text{SINGLE}}| = |\text{TEAMS}| \cdot |\text{CHAMPIONS}| = 2 \cdot 124 = 248\] In each match, only $2 \cdot 5 = 10$ of those features appear.


\item[{$\phi_{\text{PAIR}}$}]\hfill

There are $|\text{CHAMPIONS}| \cdot (|\text{CHAMPIONS}|-1)$ different 2-combinations of champions. Therefore:
    \[|\phi_{\text{PAIR}}| = |\text{TEAMS}| \cdot |\text{CHAMPIONS}| \cdot (|\text{CHAMPIONS}|-1) / 2 = 2 \cdot 124 \cdot 123 / 2 = 15252\] In each match, only $2 \cdot 5 \cdot 4 / 2 = 20$ of those features appear.


\item[{$\phi_{\text{COUNTER}}$}]\hfill

 Both teams can choose from $|\text{CHAMPIONS}|$ different champions. Therefore:
\[|\phi_{\text{COUNTER}}| = |\text{CHAMPIONS}|^2 = 124^2 = 15376\] 
In each match, only $5 \cdot 5 = 25$ of those features appear.


\item[{$\phi_{\text{BEST-RANK}}$}]\hfill

 In each match, the best ranked team can either be blue, purple, or it can be a tie. Therefore:
\[|\phi_{\text{BEST-RANK}}| = 3\] 
Each match contains only $1$ of those features since either blue, purple, or none of the teams have a the greatest rank.


\item[{$\phi_{\text{PLAYER-RANK}}$}]\hfill

Each of the $2$ teams have $5$ players that each can have $1$ of $8$ ranks:
\[|\phi_{\text{PLAYER-RANK}}| = 2 \cdot 5 \cdot 8 = 80\] 
Since every player has only $1$ rank, only $10$ of those features appear.


\item[{$\phi_{\text{LANE-RANKS}}$}]\hfill

 On each team, either \textit{none}, \textit{1}, \textit{2}, or \textit{many} champions play each of the $|\text{LANES}|$ lanes. For a given team and lane, the exact ranks are only considered if the team has either 1 or 2 champions in that. Therefore:
\[|\phi_{\text{LANE-RANKS}}| = |\text{LANES}| \cdot (1 + |\text{RANKS}| + |\text{RANKS}|^2 + 1)^2 = 21904\]
In each match only $5$ of these features appear, as each of the lanes has exactly one combination of ranks of the players / champions playing against each other.


\item[{$\phi_{\text{LANE-CHAMPION}}$}]\hfill

On each team, every champion can be in one of $|\text{LANES}|$ lanes. Therefore, we get that:
\[|\phi_{\text{LANE-CHAMPION}}| = |\text{TEAMS}| \cdot |\text{CHAMPIONS}| \cdot |L| = 2 \cdot 124 \cdot 4 = 992\]
In each match, only $2 \cdot 5 \cdot 4 = 40$ features appear.


\item[{$\phi_{\text{CHAMPION-SPELLS}}$}]\hfill

With $|\text{TEAMS}|$ teams, $|\text{CHAMPIONS}|$ champions and $|\text{SPELLS}|$ spells. Therefore, we get that:
\[|\phi_{\text{CHAMPION-SPELLS}}| = |\text{TEAMS}| \cdot |\text{CHAMPIONS}| \cdot |\text{SPELLS}| = 2 \cdot 124 \cdot 22 = 5456\]
Since each of the 10 summoners have 2 spells, $2 \cdot 10 = 20$ features appear in each match.


\item[{$\phi_{\text{CHAMPION-RUNES}}$}]\hfill

With $\text{TEAMS}$ teams, $|\text{CHAMPIONS}|$ champions and $|\text{RUNES}|$ runes. Therefore, we get that:
\[|\phi_{\text{CHAMPION-RUNES}}| = |\text{TEAMS}| \cdot |\text{CHAMPIONS}| \cdot |\text{RUNES}| = 2 \cdot 124 \cdot 296 = 73408\]
Since each of the 10 summoners have between 0 and 30 runes, the number of features that appear in each match is between
$10 \cdot 0 = 0$ and $10 \cdot 30 = 300$.


\item[{$\phi_{\text{CHAMPION-MASTERIES}}$}]\hfill

With $\text{TEAMS}$ teams, $|\text{CHAMPIONS}|$ champions and $|\text{MASTERIES}|$ masteries. Therefore, we get that:
\[|\phi_{\text{CHAMPION-MASTERIES}}| = |\text{TEAMS}| \cdot |\text{CHAMPIONS}| \cdot |\text{SPELLS}| = 2 \cdot 124 \cdot 57 = 14136\]
Since each of the 10 summoners have between 0 and 30 masteries, the number of features that appear in each match is between
$10 \cdot 0 = 0$ and $10 \cdot 30 = 300$.
\end{description}



%\paragraph{$\phi_{\text{SINGLE}}$}
%\begin{adjustwidth}{2cm}{}
%    Each team can choose from $|\text{CHAMPIONS}|$ different champions. Therefore:
%    \[|\text{TEAMS}| \cdot |\text{CHAMPIONS}| = 2 \cdot 124 = 248\] In each match, only $2 \cdot 5 = 10$ of those features appear.
%\end{adjustwidth}

%\paragraph{$\phi_{\text{PAIR}}$}
%\begin{adjustwidth}{2cm}{}
%    There are $|\text{CHAMPIONS}| \cdot (|\text{CHAMPIONS}|-1)$ different 2-combinations of champions. Therefore:
%    \[|\text{TEAMS}| \cdot |\text{CHAMPIONS}| \cdot (|\text{CHAMPIONS}|-1) / 2 = 2 \cdot 124 \cdot 123 / 2 = 15252\] In each match, only $2 \cdot 5 \cdot 4 / 2 = 20$ of those features appear.
%\end{adjustwidth}

%\paragraph{$\phi_{\text{COUNTER}}$}
%\begin{adjustwidth}{2cm}{}
%Both teams can choose from $|\text{CHAMPIONS}|$ different champions. Therefore:
%\[|\text{CHAMPIONS}|^2 = 124^2 = 15376\] 
%In each match, only $5 \cdot 5 = 25$ of those features appear.
%\end{adjustwidth}

%\paragraph{$\phi_{\text{BEST-RANK}}$}
%\begin{adjustwidth}{2cm}{}
%In each match, the best ranked team can either be blue, purple, or it can be a tie. Therefore:
%\[\phi_{\text{BEST-RANK}} = 3\] 
%Each match contains only $1$ of those features since either blue, purple, or none of the teams have a the greatest rank.
%\end{adjustwidth}

%\paragraph{$\phi_{\text{PLAYER-RANK}}$}
%\begin{adjustwidth}{2cm}{}
%Each of the $2$ teams have $5$ players that each can have $1$ of $8$ ranks:
%\[ 2 \cdot 5 \cdot 8 = 80\] 
%Since every player has only $1$ rank, only $10$ of those features appear.
%\end{adjustwidth}

%\paragraph{$\phi_{\text{LANE-RANKS}}$}
%\begin{adjustwidth}{2cm}{}
%Each lane contains either none, 1, 2, or many champions from each team. The exact ranks are only considered for a given team and lane if the team has either 1 or 2 champions in a given lane. Therefore:
%\[|\text{LANES}| \cdot (1 + |\text{RANKS}| + |\text{RANKS}|^2 + 1)^2 = 21904\]
%In each match only $5$ of these features appear, as each of the lanes has exactly one combination of ranks of the players / champions playing against each other.
%\end{adjustwidth}

%\paragraph{$\phi_{\text{LANE-CHAMPION}}$}
%\begin{adjustwidth}{2cm}{}
%On each team, every champion can be in one of 4 lanes. Therefore, we get that:
%\[|\text{TEAMS}| \cdot |\text{CHAMPIONS}| \cdot |L| = 2 \cdot 124 \cdot 4 = 992\]
%In each match, only $2 \cdot 5 \cdot 4 = 40$ features appear.
%\end{adjustwidth}

%\paragraph{$\phi_{\text{CHAMPION-SPELLS}}$}
%\begin{adjustwidth}{2cm}{}
%There exist $\text{TEAMS}$ teams, $|\text{CHAMPIONS}|$ champions and %$|\text{SPELLS}|$ spells. Therefore, we get that:
%\[|\text{TEAMS}| \cdot |\text{CHAMPIONS}| \cdot |\text{SPELLS}| = 2 \cdot 124 \cdot 22 = 5456\]
%Since each of the 10 summoners have 2 spells, $2 \cdot 10 = 20$ features appear in each match.
%\end{adjustwidth}

%\paragraph{$\phi_{\text{CHAMPION-RUNES}}$}
%\begin{adjustwidth}{2cm}{}
%There exist $\text{TEAMS}$ teams, $|\text{CHAMPIONS}|$ champions and %$|\text{RUNES}|$ runes. Therefore, we get that:
%\[|\text{TEAMS}| \cdot |\text{CHAMPIONS}| \cdot |\text{RUNES}| = 2 \cdot 124 \cdot 296 = 73408\]
%Since each of the 10 summoners have at most 30 runes, at most $10 \cdot 30 = 300$ features appear in each match.
%\end{adjustwidth}

%\paragraph{$\phi_{\text{CHAMPION-MASTERIES}}$}
%\begin{adjustwidth}{2cm}{}
%There exists $\text{TEAMS}$ teams, $|\text{CHAMPIONS}|$ champions and %$|\text{MASTERIES}|$ masteries. Therefore, we get that:
%\[|\text{TEAMS}| \cdot |\text{CHAMPIONS}| \cdot |\text{SPELLS}| = 2 \cdot 124 \cdot 57 = 14136\]
%Since each of the 10 summoners have chosen at most 30 masteries, at most $10 \cdot 30 = 300$ features appear in each match.
%\end{adjustwidth}
\begin{table}[!htb]
  \centering
  \scalebox{0.9}{
    \begin{tabular}{|l|ccc|}
      \hline
      Feature type                       & Domain size & Appears in a single match & Appeared in 60,000 matches \\ \hline
      $\phi_{\text{SINGLE}}$             & 248         & 10    & 248               \\ 
      $\phi_{\text{PAIR}}$               & 15252       & 20    & 15252             \\ 
      $\phi_{\text{COUNTER}}$            & 15376       & 25    & 15359             \\ 
      $\phi_{\text{BEST-RANK}}$          & 3           & 1     & 3                 \\ 
      $\phi_{\text{PLAYER-RANK}}$        & 80          & 10    & 80                \\ 
      $\phi_{\text{LANE-RANKS}}$         & 21904       & 5     & 1834              \\ 
      $\phi_{\text{LANE-CHAMPION}}$      & 992         & 40    & 992               \\
      $\phi_{\text{CHAMPION-SPELLS}}$    & 5456        & 20    & 2434              \\
      $\phi_{\text{CHAMPION-RUNES}}$     & 73408       & 0-300 & 52212             \\
      $\phi_{\text{CHAMPION-MASTERIES}}$ & 14136       & 0-300 & 14120             \\
      Total                              & 146855      & 131-731  & 102534           \\ \hline
    \end{tabular}
  }
  \caption{The sparsity of each type of feature. (*)At most.}\label{tab:featuresparsity}
\end{table}

The calculated sparsities are shown in \Cref{tab:featuresparsity}, which also includes how many of each type of feature that appeared in 60,000 matches.

In total, at least $0.09 \%$ and at most $0.40\%$ of all possible features appear in a single game. Given a such level of sparsity, we choose a sparse representation of features, as this reduces the memory use significantly and allows us to use feature sets of a much larger size. This allows us to fit much larger feature sets into a limited amount of memory.