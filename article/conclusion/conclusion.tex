\section{Conclusion}\label{sec:conclusion}
To work with the large dataset of 456GB comprising of 2 million matches of played league of legends games, we have set up a cluster which is able to handle a single run of Logistic Regression in about 10 hours, on semi old hardware. We have performed several experiments on this cluster and compiled an interesting list of knowledge related to the game.

The largest accuracy which we have been able to get is 58.5\% which as discussed in \Cref{sub:knowledge} is a consequence of how well the game is balanced and kept balanced by patch. One might ask if this accuracy is even helpful for a player playing the game. In a sense if a player knows the opponents' 5 champions the player might be able to utilize some champion which counters this team enormously.

To help players pick a winning combination of champions we trained a model using logistic regression on the cluster. The model was trained on almost 2 million matches using 70\% for training and 30\% as test samples. After running for a long duration, the test sample gave an accuracy of 58.5\%, while tests on the training data gave an accuracy of 58.73\%, which shows that we have no overfitting. We believe that the accuracy shows that LoL is a rather well balanced game that favours the skills of the individual players rather than the power of champion selection. In addition the method of matching players also adds an extra layer of balance as players of roughly the same skill level, will be playing against each other.



\todo{mere}