\section{Cluster}\label{sec:cluster}
In this section, the cluster that will be doing the computations will be outlined. In \Cref{sec:clustersetup} the setup of the cluster and the available resources are listed, followed by a profile of the resource usage in \Cref{sec:profile}. In \Cref{sec:speedup} the speedup from using additional nodes for computations will be measured, both using several nodes and computation speed.

\subsection{Cluster Setup}\label{sec:clustersetup}

The cluster used for this work consists of four nodes, one master and three worker nodes. We use the same master node for cluster management on Spark and storage management on HDFS.\ Some of the common fault tolerance functionality of Hadoop and Spark has not been employed due to the limited resources of the project and small size of the cluster and dataset. Most significantly, only one replication of data exists across HDFS, which naturally put large parts of the data at risk. However, given the relatively small amount of machines and ease of access to new match data, we chose to prioritise data volume over fault tolerance. We have two types of machines seen in \Cref{tab:setups}.
\begin{table}[!htb]
  \centering
  \begin{tabular}{|r|ccc|}
    \hline
      & CPU & Storage & Memory \\\hline
    Setup 1 & Dual core intel e8400 3Ghz & 220 GB 7200 RPM & 4GB DDR2 \\
    Setup 2 & Quad core intel q9400 2.66Ghz & 220 GB 7200 RPM & 8GB DDR2 \\\hline
  \end{tabular}
  \caption{The different node setups}\label{tab:setups}
\end{table}

Where the master and one worker node has a dual core and the two remaining worker nodes have the quad cores. The machines are linked together using a switch capable of 200Mbit pr port, which sets a potential one-way bandwidth of 12.5MB/s pr machine. The physical setup of the cluster can be seen in \Cref{fig:clustersetup}.

\begin{figure}[!htb]
  \centering
    \includegraphics[width=1\textwidth]{img/cluster2.jpg}
  \caption{Physical cluster setup}\label{fig:clustersetup}
\end{figure}

%%% Local Variables:
%%% mode: latex
%%% TeX-master: "../main"
%%% End:
