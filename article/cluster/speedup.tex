\subsection{Cluster speedup}\label{sec:speedup}
In this section we measure the speedup when adding additional computing nodes to the cluster.

To test the speedup 50GB match data were used to train a logistic regression model using stochastic gradient descent and L2 regularization with a ridge value of 0.01 the features used were all pre-match features described in  \Cref{sec:feattest}. The following is the time taken by 1 master controling 1, 2 and 3 worker(s) respectively:

\begin{enumerate}
    \item 43m34s (2614s)
    \item 36m22s (2182s)
    \item 23m25s (1405s)
\end{enumerate}


\begin{figure}[!htb]
  \centering
  \begin{tikzpicture}[] 
    \begin{axis}[
      xlabel=Workers, 
      ylabel=Time,
      xtick={1,2,3},
      xticklabel style={anchor=near xticklabel},
      scaled x ticks=false,
      x label style={at={(axis description cs:0.5,-0.1)},anchor=north},
      legend style={at={(1.4,1.001)},
        anchor=north,legend columns=1},] 
      \addplot[color=brown] coordinates { 
        (1,2614)
        (2,2182)
        (3,1405)  
      };
	\legend{Speed}
	\end{axis} 
\end{tikzpicture}
   \caption{Cluster speed up}\label{fig:cluster-speedup}
\end{figure}
The results of test is shown in \Cref{fig:cluster-speedup} the reason these numbers are irregular is because one node only has 2 cores, while the other two nodes has 4 cores. We can measure the speedup using Amdahl's law:

\[S(N) = \frac{1}{(1-P)+\frac{P}{N}}\]

Where $N$ is the number pro processors, and $P$ is the proportion of the program that can be mad parallel. We do not know $P$ in this case, but we can estimate $P$ using the measure of 1 and 3 nodes:

\[P = \frac{\frac{1}{S_m}-1}{\frac{1}{N}-1}  \]

Where \( S_m=\frac{2614}{1405} = 1.86 \). We then get $P = 0.528$, which is quite good.