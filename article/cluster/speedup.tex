\subsection{Cluster speedup}
In this section we measure the speedup when adding additional computing nodes to the cluster.

To test the speedup we set up performed supervised learning on 50GB match data with on 200k features. The following is the time taken by 1,2 and 3 nodes respectively:

\begin{enumerate}
    \item 43m34s (2614s)
    \item 36m22s (2182s)
    \item 23m25s (1405s)
\end{enumerate}

The reason these numbers are irregular is because one node only has 2 cores, while the other two nodes has 4 cores. We can measure the speedup using Amdahl's law:

\[S(N) = \frac{1}{(1-P)+\frac{P}{N}}\]

Where $N$ is the number pro processors, and $P$ is the proportion of the program that can be mad parallel. We do not know $P$ in this case, but we can estimate $P$ using the measure of 1 and 3 nodes:

\[P = \frac{\frac{1}{S_m}-1}{\frac{1}{N}-1}  \]

Where \( S_m=\frac{2614}{1405} = 1.86 \). We then get $P = 0.528$, which is quite good ehehe.