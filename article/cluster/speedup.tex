\subsection{Cluster speedup}\label{sec:speedup}
To test the speedup, 50GB match data were used to train a logistic regression model using stochastic gradient descent and L2 regularisation with a ridge value of 0.01 the features used were all pre-match features described in \Cref{sec:choosingfeatures}. The following is the time taken by the master node controlling 1, 2 and 3 workers respectively.

\begin{enumerate}
    \item 43minutes 34seconds which is 2614seconds (2 cores)
    \item 36minutes 22seconds which is 2182seconds (6 cores)
    \item 23minutes 25seconds which is 1405seconds (8 cores)
\end{enumerate}

\begin{figure}[!htb]
  \centering
  \begin{tikzpicture}[] 
    \begin{axis}[
      xlabel=Workers, 
      ylabel=Seconds,
      xtick={1,2,3},
      xticklabel style={anchor=near xticklabel},
      scaled x ticks=false,
      x label style={at={(axis description cs:0.5,0.0)},anchor=north},
      legend style={at={(1.2,1.001)},
        anchor=north,legend columns=1},] 
      \addplot[color=brown] coordinates { 
        (1,2614)
        (2,2182)
        (3,1405)  
      };
	\legend{Speed}
	\end{axis} 
\end{tikzpicture}
   \caption{Cluster speed up}\label{fig:cluster-speedup}
\end{figure}
The results of test is shown in \Cref{fig:cluster-speedup} the reason these numbers are irregular is because one node has 4 processor, while the other two nodes has 4 processor. We can measure the speedup using Amdahl's law:

\[S(N) = \frac{1}{(1-P)+\frac{P}{N}}\]

Where $N$ is the number of processors, and $P$ is the proportion of the program that is parallel. We do not know $P$ in this case, but we can estimate $P$ by calculating the difference of speedup between node 1(2 cores) and 3 (8 cores):

\[P_{estimated} = \frac{\frac{1}{S_m}-1}{\frac{1}{N}-1}  \]

Where \( S_m=\frac{43m34s}{23m25s} = 1.86 \). We then get $P = 0.528$. Using this $P$ value we can estimate the maximal number of processors worth using, this is shown in \Cref{fig:speedupcon}. The plot shows that our speedup converges at around 2 using between 40 and 80 processors. This means that adding more processors will help only marginally more.

\begin{figure}[!htb]
  \centering
  \begin{tikzpicture}[] 
    \begin{axis}[
      xlabel=Processors, 
      ylabel=Speedup,
      xtick={1,10,20,30,40,50,60,70,80,90},
      xticklabel style={anchor=near xticklabel},
      scaled x ticks=false,
      x label style={at={(axis description cs:0.5,0.0)},anchor=north},
      legend style={at={(1.325,1.001)},
        anchor=north,legend columns=1},] 
      \addplot[color=brown] coordinates { 
(1.0,1.0)
(2,1.358695652)
(3,1.543209877)
(4,1.655629139)
(5,1.731301939)
(6,1.785714286)
(7,1.826722338)
(8,1.858736059)
(9,1.884422111)
(10,1.905487805)
(11,1.923076923)
(12,1.937984496)
(13,1.950780312)
(14,1.961883408)
(15,1.971608833)
(16,1.98019802)
(17,1.987839102)
(18,1.994680851)
(19,2.00084246)
(20,2.006420546)
(21,2.011494253)
(22,2.016129032)
(23,2.02037948)
(24,2.024291498)
(25,2.027903958)
(26,2.03125)
(27,2.034358047)
(28,2.037252619)
(29,2.03995498)
(30,2.04248366)
(31,2.044854881)
(32,2.047082907)
(33,2.049180328)
(34,2.051158301)
(35,2.053026748)
(36,2.054794521)
(37,2.056469542)
(38,2.058058925)
(39,2.059569075)
(40,2.061005771)
(41,2.062374245)
(42,2.063679245)
(43,2.064925086)
(44,2.066115702)
(45,2.067254686)
(46,2.068345324)
(47,2.069390631)
(48,2.070393375)
(49,2.071356104)
(50,2.072281167)
(51,2.073170732)
(52,2.074026803)
(53,2.074851237)
(54,2.075645756)
(55,2.07641196)
(56,2.077151335)
(57,2.077865267)
(58,2.078555046)
(59,2.079221878)
(60,2.079866889)
(61,2.080491132)
(62,2.081095596)
(63,2.081681205)
(64,2.082248829)
(65,2.082799282)
(66,2.083333333)
(67,2.083851704)
(68,2.084355076)
(69,2.08484409)
(70,2.085319352)
(71,2.085781434)
(72,2.086230876)
(73,2.086668191)
(74,2.087093863)
(75,2.08750835)
(76,2.087912088)
(77,2.088305489)
(78,2.088688946)
(79,2.089062831)
(80,2.089427497)
(81,2.089783282)
(82,2.090130506)
(83,2.090469474)
(84,2.090800478)
(85,2.091123795)
(86,2.091439689)
(87,2.091748413)
(88,2.092050209)
(89,2.092345308)
(90,2.092633929)
    };
  \legend{Speedup curve}
  \end{axis} 
\end{tikzpicture}
   \caption{Speedup convergence}\label{fig:speedupcon}
\end{figure}


%%% Local Variables:
%%% mode: latex
%%% TeX-master: "../main"
%%% End:
