\subsection{Cluster speedup}\label{sec:speedup}
In this section we measure the speedup when adding additional computing nodes to the cluster.

To test the speedup 50GB match data were used to train a logistic regression model using stochastic gradient descent and L2 regularization with a ridge value of 0.01 the features used were all pre-match features described in  \Cref{sec:feattest}. The following is the time taken by 1 master controling 1, 2 and 3 worker(s) respectively:

\begin{enumerate}
    \item 43m34s (2614s)
    \item 36m22s (2182s)
    \item 23m25s (1405s)
\end{enumerate}


\begin{figure}[!htb]
  \centering
  \begin{tikzpicture}[] 
    \begin{axis}[
      xlabel=Workers, 
      ylabel=Time,
      xtick={1,2,3},
      xticklabel style={anchor=near xticklabel},
      scaled x ticks=false,
      x label style={at={(axis description cs:0.5,-0.1)},anchor=north},
      legend style={at={(1.4,1.001)},
        anchor=north,legend columns=1},] 
      \addplot[color=brown] coordinates { 
        (1,2614)
        (2,2182)
        (3,1405)  
      };
	\legend{Speed}
	\end{axis} 
\end{tikzpicture}
   \caption{Cluster speed up}\label{fig:cluster-speedup}
\end{figure}
The results of test is shown in \Cref{fig:cluster-speedup} the reason these numbers are irregular is because one node only has 2 processor, while the other two nodes has 4 processor. We can measure the speedup using Amdahl's law:

\[S(N) = \frac{1}{(1-P)+\frac{P}{N}}\]

Where $N$ is the number pro processors, and $P$ is the proportion of the program that can be mad parallel. We do not know $P$ in this case, but we can estimate $P$ using the measure of 1 and 3 nodes:

\[P = \frac{\frac{1}{S_m}-1}{\frac{1}{N}-1}  \]

Where \( S_m=\frac{2614}{1405} = 1.86 \). We then get $P = 0.528$. Using this $P$ value we can estimate the maximal number of processors worth using, this is shown in \Cref{fig:speedupcon}. The plot shows that our speedup converges at around 2 using between 40 and 80 processors. This means that adding more processors will help only marginally more.


\begin{figure}[!htb]
  \centering
  \begin{tikzpicture}[] 
    \begin{axis}[
      xlabel=Speedup, 
      ylabel=Processors,
      xtick={1,1.2,1.4,1.6,1.8,2,2.2,3},
      xticklabel style={anchor=near xticklabel},
      scaled x ticks=false,
      x label style={at={(axis description cs:0.5,-0.1)},anchor=north},
      legend style={at={(1.4,1.001)},
        anchor=north,legend columns=1},] 
      \addplot[color=brown] coordinates { 
(1.0,1.0)
(1.358695652,2)
(1.543209877,3)
(1.655629139,4)
(1.731301939,5)
(1.785714286,6)
(1.826722338,7)
(1.858736059,8)
(1.884422111,9)
(1.905487805,10)
(1.923076923,11)
(1.937984496,12)
(1.950780312,13)
(1.961883408,14)
(1.971608833,15)
(1.98019802,16)
(1.987839102,17)
(1.994680851,18)
(2.00084246,19)
(2.006420546,20)
(2.011494253,21)
(2.016129032,22)
(2.02037948,23)
(2.024291498,24)
(2.027903958,25)
(2.03125,26)
(2.034358047,27)
(2.037252619,28)
(2.03995498,29)
(2.04248366,30)
(2.044854881,31)
(2.047082907,32)
(2.049180328,33)
(2.051158301,34)
(2.053026748,35)
(2.054794521,36)
(2.056469542,37)
(2.058058925,38)
(2.059569075,39)
(2.061005771,40)
(2.062374245,41)
(2.063679245,42)
(2.064925086,43)
(2.066115702,44)
(2.067254686,45)
(2.068345324,46)
(2.069390631,47)
(2.070393375,48)
(2.071356104,49)
(2.072281167,50)
(2.073170732,51)
(2.074026803,52)
(2.074851237,53)
(2.075645756,54)
(2.07641196,55)
(2.077151335,56)
(2.077865267,57)
(2.078555046,58)
(2.079221878,59)
(2.079866889,60)
(2.080491132,61)
(2.081095596,62)
(2.081681205,63)
(2.082248829,64)
(2.082799282,65)
(2.083333333,66)
(2.083851704,67)
(2.084355076,68)
(2.08484409,69)
(2.085319352,70)
(2.085781434,71)
(2.086230876,72)
(2.086668191,73)
(2.087093863,74)
(2.08750835,75)
(2.087912088,76)
(2.088305489,77)
(2.088688946,78)
(2.089062831,79)
(2.089427497,80)
(2.089783282,81)
(2.090130506,82)
(2.090469474,83)
(2.090800478,84)
(2.091123795,85)
(2.091439689,86)
(2.091748413,87)
(2.092050209,88)
(2.092345308,89)
(2.092633929,90)
      };
  \legend{Speedup curve}
  \end{axis} 
\end{tikzpicture}
   \caption{Speedup convergence}\label{fig:speedupcon}
\end{figure}