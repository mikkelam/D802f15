\subsection{Performance of the cluster}\label{sec:benchmark}
Before using the cluster to perform machine learning operations on large amounts of data, a benchmark comparison with a standalone computer will give a good idea of the increase in computing power gained.

To make the comparison three different files of the sizes 1GB, 5GB, and 10GB.\@ will be generated. They contain unsorted random numbers between $-2147483646$ and $2147483647$. To test performance a word count is performed on the cluster with the specifications seen in \Cref{sec:clustersetup}, and on a computer with:
\begin{itemize}
\item Quad core Intel(R) i7-4500U @ 1.80Ghz 2.40Ghz
\item 8GB RAM
\end{itemize}

In \Cref{tab:bench} the time used to perform word count in seconds can be seen, it shows that the cluster can count much faster than a regular standalone computer and this result also makes the cluster comparable to other such distributed systems.

\begin{table}[!htb]
  \centering
  \begin{tabular}{|c|ll|}
    \hline
    File size & Cluster  & Standalone \\
    \hline
    1GB & ? & 1651.61seconds \\
    5GB & ? & ? \\
    10GB & ? & ? \\
    \hline
  \end{tabular}
  \caption{Benchmark tests for the cluster}
  \label{tab:bench}
\end{table}




%%% Local Variables:
%%% mode: latex
%%% TeX-master: "../main"
%%% End:
