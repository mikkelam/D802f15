\subsection{Profiling resource consumption}\label{sec:profile}
%\todo{læs og udkommenter figref på linje 2}
To understand the load on the cluster's resources, we use a monitoring tool to measure the resource consumption of the nodes, including main memory, CPU load, and transfer speeds across the network and disks. We see that the majority of bottlenecking appears on the worker nodes, while the logistic regression model is being trained. In 
\Cref{fig:profilegraph} 
we can see four graphs with an extract of these measurements. The graphs are aligned over the X-axis which spans 1,677 seconds, or what is effectively two iterations of Spark's logistic regression, as described in \Cref{sec:sparklogreg}. Looking at one of the iterations, we first see that the CPU bottlenecks the system, while the disk continuously read new data into main memory, keeping as much data ready for computations as possible. The resource consumption of this phase is caused by map procedures. The second phase is bottlenecked by the transfer speeds of the network, which is caused by reduce operations. The short spike in network transfer rate presumably exist, as for at few seconds the measured node $A$ is the only one that sends data to the single reduce worker $B$. The transfer rate on the measured node is halved, as a third worker $C$ begins to transfer data to node $B$ as well. In other words, the transfer speed is bottlenecked on the $Switch \rightarrow B$ connection, while the rate from $A \rightarrow Switch$ and $C \rightarrow Switch$ each transfer with half of the $Switch \rightarrow B$ connection's maximum rate.

The measurements shows that the overall job execution can be sped up by improving either the processing or network speeds of the cluster. This observation is not that surprising, as logistic regression with stochastic gradient descent is a relatively expensive computation. Other strategies to increase job execution speeds, include a more advanced parsing technique, such that the entire JSON object is not parsed into memory, reducing the computation overhead. In the same manner, it is also possible to split match data across multiple files. For instance the pre- and mid-game data can be split into separate files, allowing pre-selection of only the needed pieces of data, which again reduce the overhead of loading and parsing unnecessary data.

As a final note we can observe that compression of match data, would probably only increase the job execution time, as the cluster is already computationally bottlenecked. The JSON data is however very compressible, as there is a large amount of recurring keys and values in the JSON objects. As an example 1000 matches of JSON has the size of 157 MB, which can be compressed to 35MB, 22\% of its original size, with the Hadoop supported compression algorithm LZ4. The very fast LZ4 algorithm is normally used at the shuffle step of MapReduce, to improve the transfer time of intermediate data to reducer-workers. However no compression has been used on the cluster and the benefits of using a format like LZ4 with logistic regression is unknown.

\begin{figure}

\begin{tikzpicture}
	%\pgfplotsset{every axis legend/.append style={ 
		%at={(0.5,1.1)},
		%anchor=south}}
	\begin{axis}[
			enlargelimits=false,
			width=1.0\linewidth,
			height=5cm,
			stack plots=y,
			area style,
			%scaled ticks = true,
			%scaled y ticks=base 10:-4,
			xlabel = Seconds,
            	ylabel = CPU usage (\%),
			ymin = 0, 	
			ymax = 120,
            	%legend columns=-1,
            	legend entries={CPU},
			%legend style={/tikz/every even column/.append style={column sep=0.3cm}}
			]
		\addplot+table[x=Sec, y=CPUUser, col sep=space]
		{cluster/mid.csv} \closedcycle;
%		\addlegendentry{\textbf{BW}}
	\end{axis}
\end{tikzpicture}



\begin{tikzpicture}
	%\pgfplotsset{every axis legend/.append style={ 
		%at={(0.5,1.1)},
		%anchor=south}}
	\begin{axis}[
			enlargelimits=false,
			width=1.0\linewidth,
			height=5cm,
			stack plots=y,
			area style,
			%scaled ticks = true,
			%scaled y ticks=base 10:-4,
			xlabel = Seconds,
            	ylabel = Disk Transfer (KB),
			%ymin = 0, 	
			%ymax = 220,
            	%legend columns=-1,
            	legend entries={Read, Write},
			%legend style={/tikz/every even column/.append style={column sep=0.3cm}}
			]
		\addplot+table[x=Sec, y=DSKReadTot, col sep=space]
		{cluster/mid.csv} \closedcycle;
		\addplot+table[x=Sec, y=DSKWriteTot, col sep=space]
		{cluster/mid.csv} \closedcycle;
%		\addlegendentry{\textbf{BW}}
	\end{axis}
\end{tikzpicture}

\begin{tikzpicture}
	%\pgfplotsset{every axis legend/.append style={ 
		%at={(0.5,1.1)},
		%anchor=south}}
	\begin{axis}[
			enlargelimits=false,
			width=1.0\linewidth,
			height=5cm,
			stack plots=y,
			area style,
			%scaled ticks = true,
			%scaled y ticks=base 10:-4,
			xlabel = Seconds,
            	ylabel = Network Transfer (KB),
			%ymin = 0, 	
			%ymax = 15000,
            	%legend columns=-1,
            	legend entries={Transmit, Recieve},
			%legend style={/tikz/every even column/.append style={column sep=0.3cm}}
			]
		\addplot+table[x=Sec, y=NETRxKBTot, col sep=space]
		{cluster/mid.csv} \closedcycle;
		\addplot+table[x=Sec, y=NETTxKBTot, col sep=space]
		{cluster/mid.csv} \closedcycle;
%		\addlegendentry{\textbf{BW}}
	\end{axis}
\end{tikzpicture}

\begin{tikzpicture}
	%\pgfplotsset{every axis legend/.append style={ 
		%at={(0.5,1.1)},
		%anchor=south}}
	\begin{axis}[
			enlargelimits=false,
			width=1.0\linewidth,
			height=5cm,
			stack plots=y,
			area style,
			%scaled ticks = true,
			%scaled y ticks=base 10:-4,
			xlabel = Seconds,
            	ylabel = Memory used (KB),
			ymin = 0, 	
			ymax = 8500000,
            	%legend columns=-1,
            	legend entries={Memory},
			%legend style={/tikz/every even column/.append style={column sep=0.3cm}}
			]
		\addplot+table[x=Sec, y=MEMUsed, col sep=space]
		{cluster/mid.csv} \closedcycle;
%		\addlegendentry{\textbf{BW}}
	\end{axis}
\end{tikzpicture}

\caption{caption} \label{fig:profilegraph}
\end{figure}
\todo{uncomment the graph-input}


